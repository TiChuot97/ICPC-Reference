\documentclass[10pt,landscape]{article}
\usepackage{multicol}
\usepackage{listings}
\usepackage{color}
\usepackage[margin=0.75in]{geometry}
\usepackage{titling}
\setlength{\droptitle}{2in}
\definecolor{dkgreen}{rgb}{0,0.6,0}
\definecolor{gray}{rgb}{0.5,0.5,0.5}
\definecolor{mauve}{rgb}{0.58,0,0.82}
\usepackage{rotating}
\usepackage{graphics}
\usepackage{fancyhdr}
\pagestyle{fancy}

\lhead{University of Illinois at Urbana-Champaign}
\rhead{VIM - Help poor children!}

\lstset{frame=tb,
		language=C++,
		aboveskip=2mm,
		belowskip=2mm,
		showstringspaces=false,
		columns=flexible,
		basicstyle={\small\ttfamily},
		numbers=none,
		numberstyle=\tiny\color{gray},
		keywordstyle=\color{dkgreen},
		commentstyle=\color{mauve},
		breaklines=true,
		breakatwhitespace=true,
		tabsize=2
}


\begin{document}

\begin{multicols}{2}

\setcounter{page}{1}
\pagenumbering{arabic}
\pagestyle{fancy}
\tableofcontents
\newpage

\section{Getting Started}
\subsection{Vimrc}
\begin{lstlisting}
syntax on
set nu
set ruler
set autoindent
set smartindent
set expandtab
set tabstop=4
set shiftwidth=4
\end{lstlisting}
\subsection{Starter Code}
\begin{lstlisting}
#include <bits/stdc++.h>
#define LL long long
#define F first
#define S second

using namespace std;

int main () {
  cin.tie(0);
  ios_base::sync_with_stdio(0);
  
  return 0;
}
\end{lstlisting}
\subsection{C++ Grammar, STL}
\begin{lstlisting}
string s; getline(cin, s);  // read one line
stringstream ss(s); int a; ss >> a; ss.ignore();  // read comma-separated integers

bool valid = next_permutation(b, e);
bool found = binary_search(b, e, val, cmp);
auto it = lower_bound(b, e, val, cmp);  // first element >= val
auto it = upper_bound(b, e, val, cmp);  // first element > val
stable_sort(b, e, cmp);  // preserve relative order of eq vals
unique(b, e);

struct Cmp { bool operator() (T &a, T &b) { return true; } };
set<T,Cmp> s;
bool cmp (T &a, T &b) { return true; }
set<T,decltype(cmp)> s(cmp);
auto cmp = [](T &a, T &b) -> bool { return true; }
set<T,decltype(cmp)> s(cmp);

map<int,int> m;
m.find(val) == m.end()
for (auto p : m) { key = p.F; value = p.S; }

priority_queue<T, vector<T>, Cmp> pq;
\end{lstlisting}
\section{Data Structures}
\subsection{Binary Indexed Tree 2D}
\begin{lstlisting}
// Support 2 types of queries:
// - Add v to cell (x, y)
// - Get the sum of rectangle with top-left corner (1, 1) 
// and lower-right corner (x, y)

void update(int x, int y, int v) {
    while (x <= n) {
        int z = y;
        while (z <= n) {
            bit[x][z] += v;
            z += (z & (-z));
        }
        x += (x & (-x));
    }
}

int get(int x, int y) {
    if (x == 0 || y == 0) return 0;
    int sum = 0;
    while (x) {
        int z = y;
        while (z) {
            sum += bit[x][z];
            z -= (z & (-z));
        }
        x -= (x & (-x));
    }
    return sum;
}
\end{lstlisting}
\subsection{Segment Tree 2D}
\begin{lstlisting}
// Supported:
// - Add a value v to cell (x, y)
// - Get the sum in rectangle with top left corner 
// (x1, y1) and bottom right corner (x2, y2)

void build_y(int k_x, int k_y, int l, int r) {
    if (l == r) { 
        t[k_x][k_y] = 0; 
        return;
    }
    int mid = (l + r) >> 1;
    build_y(k_x, k_y * 2, l, mid);
    build_y(k_x, k_y * 2 + 1, mid + 1, r);
    t[k_x][k_y] = 0;
}

void build_x(int k, int l, int r) {
    build_y(k, 1, 1, n);
    if (l == r) return;
    int mid = (l + r) >> 1;
    build_x(k * 2, l, mid);
    build_x(k * 2 + 1, mid + 1, r);
}

void update_y(int k_x, int l_x, int r_x, int k_y, int l_y, int r_y, int y, int v) {
    if (y < l_y || r_y < y) return;
    if (l_y == r_y) {
        if (l_x == r_x)
            t[k_x][k_y] += v;
        else
            t[k_x][k_y] = t[k_x * 2][k_y] + t[k_x * 2 + 1][k_y];
        return;
    }
    int mid = (l_y + r_y) >> 1;
    update_y(k_x, l_x, r_x, k_y * 2, l_y, mid, y, v);
    update_y(k_x, l_x, r_x, k_y * 2 + 1, mid + 1, r_y, y, v);
    t[k_x][k_y] = t[k_x][k_y * 2] + t[k_x][k_y * 2 + 1];
}

void update_x(int k, int l, int r, int x, int y, int v) {
    if (x < l || r < x) return;
    if (l == r) {
        update_y(k, l, r, 1, 1, n, y, v);
        return;
    }
    int mid = (l + r) >> 1;
    update_x(k * 2, l, mid, x, y, v);
    update_x(k * 2 + 1, mid + 1, r, x, y, v);
    update_y(k, l, r, 1, 1, n, y, v);
}

int get_y(int k_x, int k_y, int l, int r, int y1, int y2) {
    if (y2 < l || r < y1) return 0;
    if (y1 <= l && r <= y2) return t[k_x][k_y];
    int mid = (l + r) >> 1;
    return get_y(k_x, k_y * 2, l, mid, y1, y2) +
           get_y(k_x, k_y * 2 + 1, mid + 1, r, y1, y2); 
}

int get_x(int k, int l, int r, int x1, int x2, int y1, int y2) {
    if (r < x1 || x2 < l) return 0;
    if (x1 <= l && r <= x2)
        return get_y(k, 1, 1, n, y1, y2);
    int mid = (l + r) >> 1;
    return get_x(k * 2, l, mid, x1, x2, y1, y2) +
           get_x(k * 2 + 1, mid + 1, r, x1, x2, y1, y2); 
}
\end{lstlisting}
\subsection{Persistent Segment Tree}
\begin{lstlisting}
struct Node {
    Node() = default;

    Node(int l, int r, int v)
        : left(l), right(r), val(v) {}

    int left, right, val;
};

int build(int k, int l, int r) {
    tree[k].val = 0;
    if (l == r) return k;
    tree[k].left = ++num_node;
    tree[k].right = ++num_node;
    int mid = (l + r) >> 1;
    build(tree[k].left, l, mid);
    build(tree[k].right, mid + 1, r);
    return k;
}

int update(int k, int l, int r, int i, int v) {
    int K = ++num_node;
    if (l == r) {
        tree[K].val = tree[k].val + v;
        return K;
    }
    tree[K].left = tree[k].left;
    tree[K].right = tree[k].right;
    int mid = (l + r) >> 1;
    if (i <= mid)
        tree[K].left = update(tree[K].left, l, mid, i, v);
    else
        tree[K].right = update(tree[K].right, mid + 1, r, i, v);
    tree[K].val = tree[tree[K].left].val + tree[tree[K].right].val;
    return K;
}
\end{lstlisting}
\subsection{Treap}
\begin{lstlisting}
// Treap
// Tested on POJ 2828

struct Node {
    int v, k, size, cnt;
    Node *l, *r;
    Node (int v) : v(v), k(rand()), size(1), cnt(1), l(NULL), r(NULL) {}
    void update () {
        size = (l ? l->size : 0) + (r ? r->size : 0) + cnt;
    }
};
Node *root;

void zig (Node* &u) {
    Node *v = u->l;
    if (!v) return;
    u->l = v->r; v->r = u;
    v->size = u->size; u->update();
    u = v;
}

void zag (Node* &u) {
    Node *v = u->r;
    if (!v) return;
    u->r = v->l; v->l = u;
    v->size = u->size; u->update();
    u = v;
}

void insert (Node* &u, int v) {
    if (!u) { u = new Node(v); return; }
    if (v < u->v) {
        insert(u->l, v);
        if (u->l->k < u->k) zig(u);
    } else if (v > u->v) {
        insert(u->r, v);
        if (u->r->k < u->k) zag(u);
    } else {
        u->cnt++;
    }
    u->update();
}
\end{lstlisting}
\subsection{Splay Tree}
\begin{lstlisting}
// Supports reversing a segment.

struct SplayTree {
    struct Node {
        Node *left, *right, *parent;
        int value, size;
        bool reversed;
    };

    SplayTree() {
        nilt = new Node();
        nilt->left = nilt->right = nilt->parent = nilt;
        nilt->value = nilt->size = 0;
        nilt->reversed = false;
    }

    void set_left(Node* x, Node* y) {
        x->left = y;
        y->parent = x;
    }

    void set_right(Node* x, Node* y) {
        x->right = y;
        y->parent = x;
    }

    void set_child(Node* x, Node* y, bool is_right) {
        if (is_right) set_right(x, y);
        else set_left(x, y);
    }

    void build_tree(vector < int >& arr) {
        root = nilt;
        for (int i = 0; i < arr.size(); ++i) {
            Node* x = new Node();
            x->size = arr[i];
            x->value = arr[i];
            x->reversed = false;
            set_left(x, root);
            x->parent = x->right = nilt;
            root = x;
        }
    }

    void propagate(Node* x) {
        if (x == nilt) return;
        if (x->reversed) {
            swap(x->left, x->right);
            x->left->reversed = !x->left->reversed;
            x->right->reversed = !x->right->reversed;
            x->reversed = false;
        }
    }

    Node* locate(Node* x, int pos) {
        do {
            propagate(x);
            int num = x->left->size + 1;
            if (num == pos) return x;
            if (num > pos) x = x->left;
            else
                pos -= num, x = x->right;
        } while (true);
        return x;
    }

    void update(Node* x) {
        x->size = x->left->size + x->right->size + 1;
    }

    void uptree(Node* x) {
        Node* y = x->parent;
        Node* z = y->parent;
        if (x == y->right) {
            Node* b = x->left;
            set_right(y, b);
            set_left(x, y);
        }
        else {
            Node* b = x->right;
            set_left(y, b);
            set_right(x, y);
        }
        update(y);
        update(x);
        set_child(z, x, z->right == y);
    }

    void splay(Node* x) {
        do {
            Node* y = x->parent;
            if (y == nilt) return;
            Node* z = y->parent;
            if (z != nilt) {
                if ((x == y->left) == (y == z->left))
                    uptree(y);
                else
                    uptree(x);
            }
            uptree(x);
        } while (true);
    }

    void split(Node* t, int pos, Node*& t1, Node*& t2) {
        if (pos == 0) {
            t1 = nilt;
            t2 = t;
            return;
        }
        if (pos >= t->size) {
            t1 = t;
            t2 = nilt;
            return;
        }
        Node* x = locate(t, pos);
        splay(x);
        t1 = x;
        t2 = x->right;
        t1->right = nilt;
        t2->parent = nilt;
        update(t1);
    }

    Node* join(Node* t1, Node* t2) {
        if (t1 == nilt) return t2;
        t1 = locate(t1, t1->size);
        splay(t1); 
        set_right(t1, t2);
        update(t1);
        return t1;
    }

    Node *root, *nilt;
};
\end{lstlisting}
\subsection{Mo's Algorithm}
\begin{lstlisting}
// The array is 1-based

bool cmp_mo(Query i, Query j) {
    int s = (int) sqrt(n);
    return ((i.l - 1) / s < (j.l - 1) / s || ((i.l - 1) / s == (j.l - 1) / s && i.r < j.r));
}
\end{lstlisting}
\section{Graph Theory}
\subsection{Ford Fulkerson}
\begin{lstlisting}
bool find_path() {
    int l = 1, r = 1; ++flag;
    q[1] = source; check[source] = flag;
    while (l <= r) {
        int u = q[l++];
        for (auto v : adj[u])
            if (check[v] != flag && c[u][v] > f[u][v]) {
                pre[v] = u;
                check[v] = flag;
                if (v == target) return true;
                q[++r] = v;
            }
    }
    return false;
}

void augment() {
    int v = target, delta = oo;
    while (v != source) {
        int u = pre[v];
        delta = min(c[u][v] - f[u][v], delta);
        v = u;
    }
    v = target; flow += delta;
    while (v != source) {
        int u = pre[v];
        f[u][v] += delta;
        f[v][u] -= delta;
        v = u;
    }
}
\end{lstlisting}
\subsection{Dinic}
\begin{lstlisting}
const int MAXN = 1024;

struct Edge {
	int u, v, c;
	Edge *next, *rev;
	void set(int u, int v, int c, Edge *next, Edge *rev) {
		this->u = u;
		this->v = v;
		this->c = c;
		this->next = next;
		this->rev = rev;
	}
};

struct Node {
	Edge *head;
	int level;
	Node() : head(NULL), level(-1) {}
};

struct Graph {
	int n, m;
	Node *nodes;
	Edge *edges;

	Graph() {
		cin >> n >> m;
		nodes = new Node[n];
		edges = new Edge[2*m];
		for (int i = 0; i < m; i ++) {
			int u, v, c;
			cin >> u >> v >> c;
			edges[2*i].set(u, v, c, nodes[u].head, &edges[2*i+1]);
			nodes[u].head = &edges[2*i];
			edges[2*i+1].set(v, u, 0, nodes[v].head, &edges[2*i]);
			nodes[v].head = &edges[2*i+1];
		}
	}

	bool make_level() {
		for (int i = 0; i < n; i ++) {
			nodes[i].level = -1;
		}
		queue<Node*> queue;
		queue.push(&nodes[0]);
		nodes[0].level = 0;
		while (!queue.empty()) {
			Node* node = queue.front();
			queue.pop();
			for (Edge *edge = node->head; edge; edge = edge->next) {
				if (nodes[edge->v].level == -1 && edge->c) {
					nodes[edge->v].level = node->level + 1;
					queue.push(&nodes[edge->v]);
				}
			}
		}
		return nodes[n-1].level != -1;
	}

	int find(int u, int key) {
		if (u == n-1) return key;
		for (Edge *edge = nodes[u].head; edge; edge = edge->next) {
			if (nodes[edge->v].level == nodes[u].level + 1 && edge->c) {
				int flow = find(edge->v, min(key, edge->c));
				if (flow) {
					edge->c -= flow;
					edge->rev->c += flow;
					return flow;
				}
			}
		}
		return 0;
	}

	int dinic() {
		int ans = 0;
		int flow;
		while (make_level()) {
			while ((flow = find(0, INT_MAX))) {
				ans += flow;
			}
		}
		return ans;
	}
};
\end{lstlisting}
\subsection{Min Cost Max Flow}
\begin{lstlisting}
bool spfa()
{
    int h, t, x, y;
    rep(i, T) dis[i] = inf, at[i] = 0;
    q[t = 1] = S; dis[S] = 0; at[S] = 1;
    h = 0;
    while (h != t)
    {
        ++h; if (h > 400) h = 1;
        x = q[h];
        foredge(i, x) if (e[i].c > 0)
        {
            y = e[i].a;
            if (dis[y] > dis[x] + e[i].v)
            {
                dis[y] = dis[x] + e[i].v;
                pre[y] = i;
                if (!at[y])
                {
                    ++t; if (t > 400) t = 1;
                    q[t] = y; at[y] = 1;
                }
            }
        }
        at[x] = 0;
    }
    return dis[T] != inf;
}
int main()
{
    int ans = 0;
    while (spfa())
    {
        ans += dis[T];
        for (int x = T; x; x = e[pre[x] ^ 1].a)
        {
            e[pre[x]].c--; e[pre[x] ^ 1].c++;
        }
    }
    return 0;
}
\end{lstlisting}
\subsection{Max Matching Bipartite Graph}
\begin{lstlisting}
bool find_match(int u) {
    if (check[u] == flag) return 0;
    check[u] = flag;
    for (int v : rhs)
        if (c[u][v] && (!rmatch[v] || find_match(rmatch[v]))) {
            rmatch[v] = u;
            lmatch[u] = v;
            return true;
        }
    return false;
}

int max_matching() {
    int ret = 0;
    for (int u : rhs) rmatch[u] = 0;
    for (int u : lhs) {
        ++flag; lmatch[u] = 0;
        if (find_match(u)) ++ret;
    }
    return ret;
}
\end{lstlisting}
\subsection{Minimum Vertex Cover Bipartite Graph}
\begin{lstlisting}
void alternate(int u) {
    lmvc[u] = false;
    for (int v : rhs)
        if (c[u][v]) {
            rmvc[v] = true;
            if (rmatch[v] && lmvc[rmatch[v]])
                alternate(rmatch[v]);
        }
}

void MVC() {
    max_matching();
    for (int u : rhs) rmvc[u] = false;
    for (int u : lhs) lmvc[u] = (lmatch[u] != 0);
    for (int u : lhs)
        if (!lmvc[u]) alternate(u);
}
\end{lstlisting}
\subsection{Tarjan}
\begin{lstlisting}
// avail[] initialized to be all 0
void tarjan(int u) {
    num[u] = low[u] = ++num_node;
    st.push(u);
    for (int i = 0; i < adj[u].size(); ++i) {
        int v = adj[u][i];
        if (!avail[v]) {
            if (num[v] == 0) {
                tarjan(v);
                low[u] = min(low[u], low[v]);
            }
            else low[u] = min(low[u], num[v]);
        }
    }
    if (low[u] == num[u]) {
        int v = -1;
        ++num_comp;
        while (v != u) {
            v = st.top(); st.pop();
            comp[v] = num_comp;
            avail[v] = 1;
        }
    }
}
\end{lstlisting}
\subsection{Topo Sort}
\begin{lstlisting}
void topo_sort() {
    for (int i = 1; i <= num_comp; ++i)
        if (deg[i] == 0) q.push(i);
    int num = 0;
    while (!q.empty()) {
        int u = q.front(); q.pop();
        for (int i = 0; i < new_adj[u].size(); ++i) {
            int v = new_adj[u][i];
            --deg[v];
            if (deg[v] == 0) q.push(v);
        }
        position[u] = ++num;
    }
}
\end{lstlisting}
\subsection{2-SAT}
\begin{lstlisting}
bool two_sat() {
    for (int i = 0; i < list_node.size(); ++i)
        if (!num[list_node[i]]) tarjan(list_node[i]);
    for (int i = 0; i < list_node.size(); ++i) {
        int u = list_node[i];
        if (comp[u] == comp[neg[u]]) return false;
        for (int j = 0; j < adj[u].size(); ++j) {
            int v = adj[u][j];
            if (comp[u] == comp[v]) continue;
            new_adj[comp[u]].push_back(comp[v]);
            ++deg[comp[v]];
        }
    }
    topo_sort();
    for (int i = 0; i < list_node.size(); ++i) {
        int u = list_node[i];
        // position[u]: position of u after topo sorted
        if (position[comp[u]] > position[comp[neg[u]]])
            check[u] = 1; // Pick u (otherwise pick !u)
    }
    return true;
}
\end{lstlisting}
\subsection{Lowest Common Ancestor - $O(n\log n)$}
\begin{lstlisting}
// Note: Log = ceil(log2(n))
// d[u] = depth of node u + 1 (ie: d[root] = 1)

void buildLCA() {
    for (int i = 1; i <= n; ++i) p[i][0] = par[i];
    for (int j = 1; j <= Log; ++j)
        for (int i = 1; i <= n; ++i)
            p[i][j] = p[p[i][j - 1]][j - 1];
}

int LCA(int u, int v) {
    if (d[u] < d[v]) swap(u, v);
    for (int j = Log; j >= 0; --j)
        if (d[p[u][j]] >= d[v]) u = p[u][j];
    if (u == v) return u;
    for (int j = Log; j >= 0; --j)
        if (p[u][j] != p[v][j]) {
            u = p[u][j];
            v = p[v][j];
        }
    return p[u][0];
}
\end{lstlisting}
\subsection{Eulerian Circuit}
\begin{lstlisting}
// adj[] is unordered_map
void euler(int start) {
    stack < int > st; st.push(start);
    while (!st.empty()) {
        int u = st.top();
        if (adj[u].empty()) circuit.push_back(u), st.pop();
        else {
            auto v = adj[u].begin()->first;
            --adj[u][v]; --adj[v][u];
            if (adj[u][v] == 0) {
                adj[u].erase(v);
                adj[v].erase(u);
            }
            st.push(v);
        }
    }
}
\end{lstlisting}
\subsection{Dominator Tree}
\begin{lstlisting}
// For multitest, initialize label[], l[], r[] to 0 and
// clear adj[], radj[], bucket[], child[].

struct DominatorTree {
    DominatorTree() = default;

    bool reachable(int u) { return label[u] > 0; }

    void add_edge(int u, int v) { adj[u].push_back(v); }

    void dfs(int u) {
        label[u] = ++num; orig[num] = u;
        dsu[num] = mdsu[num] = sdom[num] = num;
        for (int v : adj[u]) {
            if (!label[v]) {
                dfs(v);
                parent[label[v]] = label[u];
            }
            radj[label[v]].push_back(label[u]);
        }
    }
     
    int find_dsu(int u, int flag = 0) {
        if (u == dsu[u]) return flag ? -1 : u;
        int p = find_dsu(dsu[u], 1);
        if (p < 0) return u;
        if (sdom[mdsu[dsu[u]]] < sdom[mdsu[u]])
            mdsu[u] = mdsu[dsu[u]];
        dsu[u] = p;
        return flag ? p : mdsu[u];
    }

    void dfs_dominator(int u) {
        l[u] = ++num;
        for (int v : child[u]) dfs_dominator(v);
        r[u] = num;
    }

    bool dominates(int u, int v) {
        return l[u] <= l[v] && l[v] <= r[u];
    }
    
    void build() {
        num = 0; dfs(root);
        for (int u = num; u >= 1; --u) {
            for (int v : radj[u])
                sdom[u] = min(sdom[u], sdom[find_dsu(v)]);
            if (u != 1) bucket[sdom[u]].push_back(u);
            for (int v : bucket[u]) {
                int w = find_dsu(v);
                if (sdom[w] == sdom[v]) idom[v] = sdom[v];
                else idom[v] = w;
            }
            if (u != 1) dsu[u] = parent[u];
        }
        for (int i = 2; i <= num; ++i) {
            if (idom[i] != sdom[i]) idom[i] = idom[idom[i]];
            child[orig[idom[i]]].push_back(orig[i]);
        }
        num = 0; dfs_dominator(root);
    }
    
    vector < int > adj[maxN], radj[maxN], bucket[maxN], child[maxN];
    int sdom[maxN], idom[maxN], dsu[maxN], mdsu[maxN], n, root, num;
    int parent[maxN], label[maxN], orig[maxN], l[maxN], r[maxN];
};
\end{lstlisting}
\subsection{Centroid Decomposition}
\begin{lstlisting}
void build(int u, int p) {
    sze[u] = 1;
    for (int v : adj[u])
        if (!elim[v] && v != p) build(v, u), sze[u] += sze[v];
}

int get_centroid(int u, int p, int num) {
    for (int v : adj[u])
        if (!elim[v] && v != p && sze[v] > num / 2)
            return get_centroid(v, u, num);
    return u;
}

void centroid_decomposition(int u) {
    build(u, -1);
    int root = get_centroid(u, -1, sze[u]);
    // Do stuffs here
    elim[root] = true;
    for (int v : adj[root])
        if (!elim[v]) centroid_decomposition(v, c + 1);
}
\end{lstlisting}
\subsection{Heavy Light Decomposition}
\begin{lstlisting}
void build(int u) {
    size_tree[u] = 1;
    for (int i = 0; i < adj[u].size(); ++i) {
        int v = adj[u][i];
        if (parent[u] == v) continue;
        parent[v] = u;
        build(v);
        size_tree[u] += size_tree[v];
    }
}

void hld(int u) {
    if (chain_head[num_chain] == 0)
        chain_head[num_chain] = u;
    chain_idx[u] = num_chain;
    arr_idx[u] = ++num_arr;
    node_arr[num_arr] = u;
    
    int heavy_child = -1;
    for (int i = 0; i < adj[u].size(); ++i) {
        int v = adj[u][i];
        if (parent[u] == v) continue;
        if (heavy_child == -1 || size_tree[v] > size_tree[heavy_child])
            heavy_child = v;
    }

    if (heavy_child != -1)
        hld(heavy_child);

    for (int i = 0; i < adj[u].size(); ++i) {
        int v = adj[u][i];
        if (v == heavy_child || parent[u] == v) continue;
        ++num_chain;
        hld(v);
    }
}

// u is an ancestor of v
int query_hld(int u, int v) {
    int uchain = chain_idx[u], vchain = chain_idx[v], ans = -1;
    while (true) {
        if (uchain == vchain) {
            get(..., arr_idx[u], arr_idx[v]);
            break;
        }
        get(..., arr_idx[chain_head[vchain]], arr_idx[v]);
        v = parent[chain_head[vchain]];
        vchain = chain_idx[v];
    }
    return ans;
}
\end{lstlisting}
\section{Dynamic Programming}
\subsection{Convex Hull Trick}
\begin{lstlisting}
// Finding max.

typedef long long htype;
typedef pair < htype, htype > line;
vector < line > lst;

bool is_bad(line l1, line l2, line l3) {
    return (1.0 * (l1.second - l2.second)) / (l2.first - l1.first) >= (1.0 * (l2.second - l3.second)) / (l3.first - l2.first);
}

// Assuming lines' slopes m are strictly increasing.
void add(htype m, htype b) {
    while (lst.size() >= 2 && is_bad(lst[lst.size() - 2], lst.back(), {m, b}))
        lst.pop_back();
    lst.push_back({m, b});
}

htype get_value(line d, htype x) {
    return d.first * x + d.second;
}

// Assuming queries' x are strictly increasing.
int pointer = 0;
htype get(htype x) {
    if (pointer > lst.size()) pointer = lst.size() - 1;
    while (pointer < lst.size() - 1 && get_value(lst[pointer], x) < get_value(lst[pointer + 1], x))
        ++pointer;
    return get_value(lst[pointer], x);
}
\end{lstlisting}
\subsection{Dynamic Convex Hull Trick}
\begin{lstlisting}
// Slow but correct. Takes O(log n) per add and query.

typedef long long htype;
// Representing a line. To query value x,
// set m = x, is_query = true.
struct Line {
    bool operator < (const Line& rhs) const {
        // Compare lines
        if (!rhs.is_query) return m < rhs.m;

        // Compare queries
        const Line* s = nxt();
        if (s == NULL) return false;
        htype x = rhs.m;
        return s->m * x + s->b > m * x + b;
    }

    htype m, b;
    bool is_query;
    mutable function < const Line*() > nxt;
};

class ConvexHullTrick : public set < Line > {
  public:
    void add(htype m, htype b) {
        auto p = insert({m, b, false});
        if (!p.second) return;
        iterator y = p.first;
        y->nxt = [=] { return (next(y) == end()) ? NULL : &(*next(y)); };
        if (is_bad(y)) {
            erase(y);
            return;
        }
        while (next(y) != end() && is_bad(next(y))) erase(next(y));
        while (y != begin() && is_bad(prev(y))) erase(prev(y));
    }

    htype get(htype x) {
        iterator y = lower_bound({x, 0, true});
        return y->m * x + y->b;
    }
  private:
    bool is_bad(iterator y) {
        iterator z = next(y);
        if (y == begin())
            return ((z == end()) ? false : y->m == z->m && y->b <= z->b);
        iterator x = prev(y);
        if (z == end())
            return (y->m == x->m && y->b <= x->b);
        return (x->b - y->b) * (z->m - y->m) >= (y->b - z->b) * (y->m - x->m);
    }
};
\end{lstlisting}
\section{String}
\subsection{Z-Function}
\begin{lstlisting}
// z[] is 1-based, z[1] = 0
void z_function(const string& s){
    int l = 0, r = 0, n = s.length();
    for (int i = 2; i <= n; ++i) {
        if (i <= r) z[i] = min(r - i + 1, z[i - l + 1]);
        else z[i] = 0;
        while (i + z[i] <= n && s[i + z[i] - 1] == s[z[i]])
            ++z[i];
        if (r < i + z[i] - 1) {
            l = i;
            r = i + z[i] - 1;
        }
    }
}
\end{lstlisting}
\subsection{Suffix Array}
\begin{lstlisting}
bool suffix_cmp(int i, int j) {
    if (pos[i] != pos[j]) return pos[i] < pos[j];
    i += gap;
    j += gap;
    return (i < N && j < N) ? pos[i] < pos[j] : i > j;
}

void build_sa() {
    N = s.size();
    for (int i = 0; i < N; ++i) sa[i] = i, pos[i] = s[i];
    for (gap = 1;; gap *= 2) {
        sort(sa, sa + N, suffix_cmp);
        for (int i = 0; i < N - 1; ++i) tmp[i + 1] = tmp[i] + suffix_cmp(sa[i], sa[i + 1]);
        for (int i = 0; i < N; ++i) pos[sa[i]] = tmp[i];
        if (tmp[N - 1] == N - 1) break;
    }
}

// height[i] = length of common prefix of suffix(sa[i]) and suffix(sa[i+1])
void build_height () {
    height.assign(n-1, -1);
    for (int i = 0, k = 0; i < n; i++) {
        if (rk[i] == n-1) continue;
        if (k) k--;
        for (int j = sa[rk[i]+1]; i+k<n && j+k<n && s[i+k] == s[j+k]; k++);
        height[rk[i]] = k;
    }
}
\end{lstlisting}
\subsection{Aho-Corasick Automata}
\begin{lstlisting}
struct Node {
	Node* next[26];
	Node* fail;
	int cnt;
	Node (Node* root) {
		memset(next, NULL, sizeof(next));
		fail = root;
		cnt = 0;
	}
};
Node* root;

void insert (string s) {
	Node* curr = root;
	for (int i = 0; i < s.length(); i++) {
		int j = s[i] - 'a';
		if (curr->next[j] == NULL) {
			curr->next[j] = new Node(root);
		}
		curr = curr->next[j];
	}
	curr->cnt++;
}

void make_fail () {
	queue<Node*> q;
	for (int i = 0; i < 26; i++) {
		if (root->next[i]) {
			q.push(root->next[i]);
		}
	}
	while (!q.empty()) {
		Node* node = q.front(); q.pop();
		for (int i = 0; i < 26; i++) {
			if (node->next[i]) {
				q.push(node->next[i]);
				Node* f = node->fail;
				while (f != root && !f->next[i]) {
					f = f->fail;
				}
				if (f->next[i]) {
					f = f->next[i];
				}
				node->next[i]->fail = f;
			}
		}
	}
}

int work (string s) {
	set<Node*> seen;
	int cnt = 0;
	Node* curr = root;
	for (int i = 0; i < s.length(); i++) {
		int j = s[i] - 'a';
		while (curr != root && !curr->next[j]) {
			curr = curr->fail;
		}
		if (curr->next[j]) {
			curr = curr->next[j];
			Node* p = curr;
			while (p != root) {
				if (seen.find(p) != seen.end()) break;
				seen.insert(p);
				cnt += p->cnt;
				p = p->fail;
			}
		}
	}
	return cnt;
}
\end{lstlisting}
\subsection{Palindromic Tree}
\begin{lstlisting}
struct Node {
    Node* next[26];  // to palindrome by extending me with a letter
    Node* sufflink;  // my LSP
    int len;  // length of this palindrome substring
    int num;  // number of palindrome substrs ending here
};
Node nodes[NMAX];
int n = 0;  // number of nodes in tree
vector<int> s;
LL ans = 0;

void build_tree () {
    nodes[0].len = -1; nodes[0].sufflink = &nodes[0];  // root 0
    nodes[1].len = 0; nodes[1].sufflink = &nodes[0];  // root 1
    n = 2;
    Node* suff = &nodes[1];  // node for LSP of processed prefix
    for (int i = 0; i < s.size(); i++) {
        // find LSP xAx
        Node* ptr = suff;
        while (1) {
            int j = i - 1 - ptr->len;
            if (j >= 0 && s[j] == s[i]) break;
            ptr = ptr->sufflink;
        }

        if (ptr->next[s[i]]) {  // palindrome substr already exists
            suff = ptr->next[s[i]];
        } else {  // add a new node
            suff = &nodes[n++];
            suff->len = ptr->len + 2;
            ptr->next[s[i]] = suff;
            if (suff->len == 1) {  // current LSP is trivial
                suff->sufflink = &nodes[1];
                suff->num = 1;
            } else {
                // find xAx's LSP xBx
                while (1) {
                    ptr = ptr->sufflink;
                    int j = i - 1 - ptr->len;
                    if (j >= 0 && s[j] == s[i]) break;
                }
                suff->sufflink = ptr->next[s[i]];
                suff->num = suff->sufflink->num + 1;
            }
        }
        ans += suff->num;
    }
}
\end{lstlisting}
\section{Game Theory}
\subsection{Nim Product}
\begin{lstlisting}
// Note: (i | j) might overflow

int nim_multiply(int x, int y) {
    int p = 0;
    for (int i = 0; i < maxLog + 1; ++i)
        if (x & (1 << i))
            for (int j = 0; j < maxLog + 1; ++j)
                if (y & (1 << j))
                    p ^= mul[i][j];
    return p;
}

void init() {
    for (int i = 0; i < maxLog + 1; ++i)
        for (int j = 0; j <= i; ++j) {
            if ((i & j) == 0) mul[i][j] = 1 << (i | j);
            else {
                mul[i][j] = 1;
                for (int t = 0; t < maxLog + 1; ++t) {
                    int k = (1 << t);
                    if (i & j & k) mul[i][j] = nim_multiply(mul[i][j], ((1 << k) * 3) >> 1);
                    else
                        if ((i | j) & k) mul[i][j] = nim_multiply(mul[i][j], (1 << k));
                }
            }
            mul[j][i] = mul[i][j];
        }
}
\end{lstlisting}
\section{Math}
\subsection{Number Theory}
\begin{lstlisting}
long long gcd (long long a, long long b) { return b == 0 ? a : gcd(b, a%b); }

long long mul_mod (long long x, long long y, long long MOD) {
  long long q = (long long)((long double)x * y / MOD);
  long long r = x * y - q * MOD;
  while (r < 0) r += MOD;
  while (r >= MOD) r -= MOD;
  return r;
}
long long pow_mod (long long b, long long e, long long MOD) {
  long long ans = 1;
  while (e) {
    if (e & 1) ans = mul_mod(ans, b, MOD);
      b = mul_mod(b, b, MOD);
      e >>= 1;
    }
  return ans;
}
\end{lstlisting}
\subsubsection{Extended Euclid}
\begin{lstlisting}
// Extended Euclid
// Solve xa + yb = gcd(a, b)
pair<long long,pair<long long,long long>> extended_euclid (long long a, long long b) {
  if (b == 0) return {a, {1, 0}};
  auto ee = extended_euclid(b, a % b);
  long long g = ee.first;
  long long y = ee.second.first;
  long long x = ee.second.second;
  y -= a / b * x;
  return {g, {x, y}};
}
\end{lstlisting}
\subsubsection{Mod Linear Equation}
\begin{lstlisting}
// Mod Linear Equation
// Solve xa = b (mod n)
// Return smallest non-negative solution. Add n/g to get all g solutions
long long mod_linear_equation (long long a, long long b, long long n) {
  auto ee = extended_euclid(a, n);
  long long g = ee.first;
  long long x = ee.second.first;
  if (b % g) return -1;
  x *= b / g;
  x %= n / g; x += n / g; x %= n / g;
  return x;
}
\end{lstlisting}
\subsubsection{Chinese Remainder Theorem}
\begin{lstlisting}
// Chinese Remainder Theorem
// Solve x = bi (mod mi)
long long chinese_remainder_theorem (vector<long long> b, vector<long long> m) {
  int n = b.size();
  long long M = 1, ans = 0;
  for (int i = 0; i < n; i++) M *= m[i];
  for (int i = 0; i < n; i++) {
    long long Mi = M / m[i];
    auto ee = extended_euclid(Mi, m[i]);
    long long xi = ee.second.first;
    ans += Mi * xi * b[i];
  }
  ans %= M; ans += M; ans %= M;
  return ans;
}
\end{lstlisting}
\subsubsection{Miller-Rabin prime test}
\begin{lstlisting}
// Miller-Rabin prime test O(log(n)^3)
// Tested on UVA 11476
bool miller_rabin (long long n, long long a) {
  if (n == 2 || n == a) return true;
  if ((n & 1) == 0) return false;
  int s = 0; long long d = n - 1; while (!(d & 1)) { d >>= 1; s++; }
  long long t = pow_mod(a, d, n);
  if (t == 1 || t == n-1) return true;
  for (; s; s--) {
    t = mul_mod(t, t, n);
    if (t == n-1) return true;
  }
  return false;
}
bool is_prime (long long n) {
  if (n < 2) return false;
  vector<int> va = {2,3,5,7,11,13,17,19,23,29,31,37};
  for (int a : va) {
    if (!miller_rabin(n, a)) return false;
  }
  return true;
}
\end{lstlisting}
\subsubsection{Pollard rho prime factorization}
\begin{lstlisting}
// Pollard rho prime factorization O(n^0.25)
// Tested on UVA 11476
long long pollard_rho (long long n) {
  // find a non-trivial prime factor of n
  // n must not be a prime (will loop forever!)
  while (1) {
    long long c = rand() % (n-1) + 1;
    long long x, y; x = y = rand() % (n-1) + 1;
    long long head = 1, tail = 2;
    while (1) {
      x = (mul_mod(x, x, n) + c) % n;
      if (x == y) break;
      auto d = gcd(abs(x-y), n);
      if (d > 1 && d < n) return d;
      if ((++head) == tail) { y = x; tail <<= 1; }
    }
  }
}
map<long long,int> factorize (long long n) {
  if (n == 1) return {};
  if (is_prime(n)) return {{n, 1}};
  map<long long,int> fac;
  auto p = pollard_rho(n);
  auto fac0 = factorize(p);
  auto fac1 = factorize(n/p);
  for (auto be : fac0) fac[be.first] += be.second;
  for (auto be : fac1) fac[be.first] += be.second;
  return fac;
}
\end{lstlisting}
\subsubsection{Primitive root}
\begin{lstlisting}
// Primitive root
// p is prime
long long primitive_root (long long p) {
  auto fac = factorize(p - 1);
  for (long long g = 1; ; g++) {
    bool ok = true;
    for (auto be : fac) {
      long long b = be.first;
      if (pow_mod(g, (p - 1) / b, p) == 1) { ok = false; break; }
    }
    if (ok) return g;
  }
  return -1; // should never reach here
}
\end{lstlisting}
\subsubsection{Discrete log}
\begin{lstlisting}
// Discrete log O(p^0.5)
// Solve a^x = b (mod p) (p is prime)
long long discrete_log (long long a, long long b, long long p) {
  long long rp = (long long)sqrt(p);
  map<long long,long long> rec;
  long long tmp = 1;
  for (long long i = 0; i < rp; i++) {
    rec[tmp] = i;
    tmp = tmp * a % p;
  }
  int cur = 1;
  for (long long q = 0; q*rp < p; q++) {
    long long r = mod_linear_equation(cur, b, p);
    if (rec.find(r) != rec.end()) return q * rp + rec[r];
    cur = cur * tmp % p;
  }
  return -1; // no solution
}
\end{lstlisting}
\subsubsection{Exp remainder}
\begin{lstlisting}
// Exp remainder O(p^0.5)
// Solve x^a = b (mod p) (p is prime)
long long exp_remainder (long long a, long long b, long long p) {
  long long g = primitive_root(p);
  long long s = discrete_log(g, b, p);
  if (b == 0) return 0;
  if (s == -1) return -1;
  auto fac = extended_euclid(a, p-1);
  long long d = fac.first;
  long long x = fac.second.first;
  long long y = fac.second.second;
  if (s % d) return -1;
  x = x * s/d;
  x %= p-1; x += p-1; x %= p-1;
  for (long long i = 0; i < d; i++) x = (x + (p-1)/d) % (p-1);
  return pow_mod(g, x, p);
}
\end{lstlisting}
\subsubsection{Euler function}
\begin{lstlisting}
// Euler function O(n^0.5)
long long phi (long long n, long long key = 2) {
  if (n == 1) return 1;
  while (n % key && key * key <= n) key++;
  if (key * key > n) return n-1;
  if (n / key % key) return phi(n/key, key+1) * (key-1);
  return phi(n/key, key) * key;
}
// Euler function preprocess O(nlogn)
void phi_gen (int n) {
  vector<int> mindiv(n+1, 0), phi(n+1, 0);
  for (int i = 1; i <= n; i++) mindiv[i] = i;
  for (int i = 2; i*i <= n; i++) {
    if (mindiv[i] != i) continue;
    for (int j = i*i; j <= n; j += i) mindiv[j] = i;
  }
  phi[1] = 1;
  for (int i = 2; i <= n; i++) {
    phi[i] = phi[i / mindiv[i]];
    if ((i / mindiv[i]) % mindiv[i] == 0) phi[i] *= mindiv[i];
    else phi[i] *= mindiv[i] - 1;
  }
}
\end{lstlisting}
\subsubsection{Mobi\"us function}
\begin{lstlisting}
// Mobius function O(n^0.5)
long long mu (long long n) {
  auto fac = factorize(n);
  for (auto be : fac) {
    if (be.second > 1) return 0;
  }
  return (fac.size() % 2 == 0) ? 1 : -1;
}
// Mobius function preprocess O(nlogn)
void mu_gen (int n) {
  vector<int> mu(n+1, 0);
  for (int i = 1; i <= n; i++) {
    int target = i == 1;
    int delta = target - mu[i];
    mu[i] = delta;
    for (int j = i+i; j <= n; j += i) mu[j] += delta;
  }
}
\end{lstlisting}
\subsection{Matrix}
\subsubsection{Matrix inverse}
\subsubsection{rref}
\subsubsection{Gaussian Elimination}
\begin{lstlisting}
// Note: ax = b
bool gaussian_elimination() {
    vector < int > row;
    for (int i = 0; i < N; ++i) row.push_back(i);
    for (int t = 0; t < N; ++t) {
        int R = -1;
        for (int i = t; i < N; ++i) {
            int r = row[i];
            if (a[r][t] > eps) {
                R = i;
                break;
            }
        }
        if (R == -1) return false;
        swap(row[R], row[t]);
        R = row[t];
        for (int i = t + 1; i < N; ++i) {
            int r = row[i];
            double p = a[r][t] / a[R][t];
            for (int c = 0; c < N; ++c)
                a[r][c] -= p * a[R][c];
            b[r] -= p * b[R];
        }
    }
    for (int i = N - 1; i >= 0; --i) {
        int r = row[i];
        for (int c = N - 1; c > i; --c)
            b[r] -= a[r][c] * res[c];
        res[r] = b[r] / a[r][i];
    }
    return true;
}
\end{lstlisting}
\subsection{Discrete Fourier Transform}
\subsubsection{Base Class}
\begin{lstlisting}
// To multiply a, b and put result in c:
// PolyMul::polynomial_multiply(a, b, c);
template < class Transform >
struct DFT {
    #define TAdd Transform::add
    #define TSub Transform::subtract
    #define TMul Transform::multiply
    typedef vector < int64_t > ivector;
    typedef typename Transform::ctype DType;
    typedef vector < DType > dvector;
    typedef vector < vector < dvector > > mdvector;
    static void init() {
        w.resize(NBIT);
        for (int iter = 0, len = 1; iter < NBIT; ++iter, len *= 2) {
            w[iter].resize(2);
            for (int invert = 0; invert < 2; ++invert) {
                w[iter][invert].assign(1 << iter, 0);
                DType wlen = Transform::generate_root(2 * len, invert);
                w[iter][invert][0] = 1;
                for (int j = 1; j < len; ++j)
                    w[iter][invert][j] = TMul(w[iter][invert][j - 1], wlen);
            }
        }
    }
    static void fft(dvector& a, bool invert = false) {
        int n = a.size();
        for (int i = 1, j = 0; i < n; ++i) {
            int bit = n >> 1; 
            for (; j & bit; bit >>= 1) j ^= bit;
            j ^= bit;
            if (j > i) swap(a[i], a[j]);
        }
        for (int iter = 0, len = 1; len < n; ++iter, len *= 2) {
            DType wlen = Transform::generate_root(2 * len, invert);
            for (int i = 0; i < n; i += 2 * len) {
                for (int j = 0; j < len; ++j) {
                    auto x = a[i + j];
                    auto y = TMul(w[iter][invert][j], a[i + j + len]);
                    a[i + j] = TAdd(x, y);
                    a[i + j + len] = TSub(x, y);
                }
            }
        }
        if (invert) Transform::invert(a);
    }
    static void polynomial_multiply(
        const ivector& a, const ivector& b, ivector& out) {
        uint32_t new_size = a.size() + b.size() - 1;
        for (NBIT = 0, N = 1; N < new_size; N *= 2, ++NBIT) {}
        dvector fa(a.begin(), a.end()), fb(b.begin(), b.end());
        fa.resize(N); fft(fa);
        fb.resize(N); fft(fb);
        for (int i = 0; i < fa.size(); ++i) fa[i] = TMul(fa[i], fb[i]);
        fft(fa, true);
        Transform::prepare_output(fa, out, new_size);
    }
    static int32_t NBIT, N;
    static mdvector w;
};
// Remember to call PolyMul::init() in main().
using PolyMul = DFT < FFT >;
template<> int32_t PolyMul::NBIT = /* max of log(n) */;
template<> int32_t PolyMul::N = 1 << PolyMul::NBIT;
template<> PolyMul::mdvector PolyMul::w = PolyMul::mdvector();
\end{lstlisting}
\subsubsection{Fast Fourier Transform}
\begin{lstlisting}
struct FFT {
    typedef vector < int64_t > ivector;
    typedef complex < double > ctype;
    typedef vector < ctype > cvector;
    static ctype add(ctype x, ctype y) { return x + y; }
    static ctype subtract(ctype x, ctype y) { return x - y; }
    static ctype multiply(ctype x, ctype y) { return x * y; }
    static ctype generate_root(int len, bool invert) {
        double alpha = 2.0 * PI / len * (invert ? -1 : 1);
        return ctype(cos(alpha), sin(alpha));
    }
    static void prepare_output(
        const cvector& vin, ivector& vout, uint32_t out_size) {
        vout.resize(out_size);
        for (int i = 0; i < out_size; ++i)
            vout[i] = llround(vin[i].real());
        while (vout.size() > 1 && vout.back() == 0)
            vout.pop_back();
    }
    static void invert(cvector& a) {
        for (auto& x : a) x /= a.size();
    }
    static double PI;
};
double FFT::PI = acos(-1.0);
\end{lstlisting}
\subsubsection{Number Theoretic Transform}
\begin{lstlisting}
struct NTT {
    typedef vector < int64_t > ivector;
    typedef int64_t ctype;
    typedef vector < ctype > cvector;

    static ctype add(ctype x, ctype y) { 
        return 1ll * x + y < mod ? x + y : x + y - mod;
    }
    static ctype subtract(ctype x, ctype y) { 
        return x < y ? 1ll * x - y + mod : x - y;
    }
    static ctype multiply(ctype x, ctype y) {
        return (1ll * x * y) % mod;
    }
    static ctype generate_root(int len, bool invert) {
        ctype wlen = invert ? inv_root : root;
        for (int i = len; i < root_pw; i <<= 1)
            wlen = (1ll * wlen * wlen) % mod;
        return wlen;
    }
    static void prepare_output(
        const cvector& vin, ivector& vout, uint32_t out_size) {
        vout = vin;
        while (vout.size() > 1 && vout.back() == 0) vout.pop_back();
    }
    static void invert(cvector& a) {
        int32_t inv_n = inverse(a.size(), mod);
        for (auto& x : a) x = (1ll * x * inv_n) % mod;
    }
    static int32_t root, inv_root, root_pw, mod;
};
// Let mod = c * 2^NBIT + 1. Then, NTT::root is
// (g^c) % mod, where g is primitive root of mod.
int32_t NTT::root = /* ... */
int32_t NTT::inv_root = inverse(NTT::root, modP);
int32_t NTT::root_pw = PolyMul::N;
int32_t NTT::mod = modP;
\end{lstlisting}
\section{Geometry}
\begin{lstlisting}
double EPS = 1e-8;
double PI = acos(-1.0);

bool equal (double x, double y) { return fabs(x - y) < EPS; }
int sign (double x) {
	if (equal(x, 0.0)) return 0;
	return x > 0.0 ? 1 : -1;
}
\end{lstlisting}
\subsection{Point}
\begin{lstlisting}
struct Point {
	double x, y;

	Point (double x, double y) : x(x), y(y) {}

	friend bool operator== (Point p, Point q) { return equal(p.x, q.x) && equal(p.y, q.y); }
	friend Point operator+ (Point p, Point q) { return Point(p.x + q.x, p.y + q.y); }
	friend Point operator- (Point p, Point q) { return Point(p.x - q.x, p.y - q.y); }
	friend Point operator* (Point p, double k) { return Point(p.x * k, p.y * k); }
	friend Point operator/ (Point p, double k) { return p * (1.0 / k); }

	static double arg (Point p) { return atan2(p.y, p.x); }
	static double norm (Point p) { return sqrt(p.x * p.x + p.y * p.y); }
	static double dot (Point p, Point q) { return p.x * q.x + p.y * q.y; }
	static double cross (Point p, Point q) { return p.x * q.y - q.x * p.y; }
	static double dist (Point p, Point q) { return norm(p - q); }
	static double det (Point p, Point q, Point r) { return cross(q-p, r-p); }
	static Point rotate (Point p, double theta) {
		return Point(p.x * cos(theta) - p.y * sin(theta), p.x * sin(theta) + p.y * cos(theta));
	}

	/* triangle */
	static Point mass_center (Point p1, Point p2, Point p3) {
		return (p1 + p2 + p3) / 3.0;
	}
	static Point outer_center (Point p1, Point p2, Point p3) {
		double a1 = p2.x - p1.x, b1 = p2.y - p1.y, c1 = (a1*a1+b1*b1) / 2.0;
		double a2 = p3.x - p1.x, b2 = p3.y - p1.y, c2 = (a2*a2+b2*b2) / 2.0;
		double d = a1 * b2 - a2 * b1;
		double x = p1.x + (c1*b2 - c2*b1) / d;
		double y = p1.y + (a1*c2 - a2*c1) / d;
		return Point(x, y);
	}
	static Point outer_center (Point p1, Point p2) {
		return (p1 + p2) / 2.0;
	}
	static Point ortho_center (Point p1, Point p2, Point p3) {
		return mass_center(p1, p2, p3) * 3.0 - outer_center(p1, p2, p3) * 2.0;
	}
	static Point inner_center (Point p1, Point p2, Point p3) {
		double a = dist(p2, p3);
		double b = dist(p3, p1);
		double c = dist(p1, p2);
		return (p1 * a + p2 * b + p3 * c) / (a + b + c);
	}
	/* triangle */

	// divide and conquer: O(nlogn)
	// tested on HDU 1007
	static pair<double,pair<Point,Point>> closest_pair (vector<Point> ps) {
		int n = ps.size();
		vector<int> rank(n);
		for (int i = 0; i < n; i++) rank[i] = i;
		sort(rank.begin(), rank.end(), [&ps](int i, int j) -> bool { return ps[i].x < ps[j].x; });
		return closest_pair(ps, rank, 0, n);
	}
	static pair<double,pair<Point,Point>> closest_pair (vector<Point> &ps, vector<int> &rank, int l, int r) {
		auto ans_cmp = [](pair<double,pair<Point,Point>> i, pair<double,pair<Point,Point>> j) -> bool { return i.first < j.first; };
		if (r - l < 20) {
			pair<double,pair<Point,Point>> ans = {0x7fffffff, {Point(0,0), Point(0,0)}};
			for (int i = l; i < r; i++) {
				for (int j = i+1; j < r; j++) {
					if (ans.first > dist(ps[rank[i]], ps[rank[j]])) {
						ans = {dist(ps[rank[i]], ps[rank[j]]), {ps[rank[i]], ps[rank[j]]}};
					}
				}
			}
			return ans;
		}
		int mid = (l + r) / 2;
		auto ans = min(closest_pair(ps, rank, l, mid), closest_pair(ps, rank, mid, r), ans_cmp);
		int tl; for (tl = l; ps[rank[tl]].x < ps[rank[mid]].x - ans.first; tl++);
		int tr; for (tr = r-1; ps[rank[tr]].x > ps[rank[mid]].x + ans.first; tr--);
		sort(rank.begin()+tl, rank.begin()+tr, [&ps](int i, int j) -> bool { return ps[i].y < ps[j].y; });
		for (int i = tl; i < tr; i++) {
			for (int j = i+1; j < min(tr, i+6); j++) {
				if (ans.first > dist(ps[rank[i]], ps[rank[j]])) {
					ans = {dist(ps[rank[i]], ps[rank[j]]), {ps[rank[i]], ps[rank[j]]}};
				}
			}
		}
		sort(rank.begin()+tl, rank.begin()+tr, [&ps](int i, int j) -> bool { return ps[i].x < ps[j].x; });
		return ans;
	}

	// farthest pair in a convex hull
	// DEBUG: maybe not good at when all points are colinear
	// tested on POJ 2187
	static pair<double,pair<Point,Point>> farthest_pair (vector<Point> ps) {
		auto ans_cmp = [](pair<double,pair<Point,Point>> i, pair<double,pair<Point,Point>> j) -> bool { return i.first < j.first; };
		int n = ps.size();
		pair<double,pair<Point,Point>> ans = {0.0, {Point(0,0), Point(0,0)}};
		if (n == 1) return ans;
		for (int i = 0, j = 1; i < n; i++) {
			while (sign(det(ps[i], ps[(i+1)%n], ps[j]) - det(ps[i], ps[(i+1)%n], ps[(j+1)%n])) == -1) {
				j = (j+1)%n;
			}
			ans = max(ans, {dist(ps[i], ps[j]), {ps[i], ps[j]}}, ans_cmp);
			ans = max(ans, {dist(ps[(i+1)%n], ps[(j+1)%n]), {ps[(i+1)%n], ps[(j+1)%n]}}, ans_cmp);
		}
		return ans;
	}

	// Graham scan: O(nlogn); result in counter-clockwise
	// tested on POJ 2187 indirectly
	static vector<Point> convex_hull (vector<Point> ps) {
		int n = ps.size();
		if (n < 3) return ps;
		for (int i = 1; i < n; i++) {
			if (ps[0].y > ps[i].y || (ps[0].y == ps[i].y && ps[0].x > ps[i].x)) {
				swap(ps[0], ps[i]);
			}
		}
		Point base = ps[0];
		sort(ps.begin()+1, ps.end(), [&](Point p, Point q) -> bool { return det(base, p, q) > 0 || (det(base, p, q) == 0 && dist(base, p) < dist(base, q)); });
		vector<Point> ans = {ps[0], ps[1], ps[2]};
		for (int i = 3; i < n; i++) {
			while (sign(det(ans[ans.size()-1], ans[ans.size()-2], ps[i])) == 1) ans.pop_back();
			ans.push_back(ps[i]);
		}
		return ans;
	}
};
\end{lstlisting}
\subsection{Line}
\begin{lstlisting}
struct Line {
	Point a, b;

	Line (Point a, Point b) : a(a), b(b) {}

	static double dist (Line l, Point p) {
		return fabs(Point::det(p, l.a, l.b) / Point::dist(l.a, l.b));
	}

	static Point proj (Line l, Point p) {
		double r = Point::dot(l.b - l.a, p - l.a) / Point::dot(l.b - l.a, l.b - l.a);
		return l.a * (1 - r) + l.b * r;
	}

	static bool on_segment (Line l, Point p) {
		return sign(Point::det(p, l.a, l.b)) == 0 && sign(Point::dot(p - l.a, p - l.b)) <= 0;
	}

	static bool parallel (Line l, Line m) {
		return sign(Point::cross(l.a - l.b, m.a - m.b)) == 0;
	}

	static Point line_x_line (Line l, Line m) {
		double s1 = Point::det(m.a, l.a, m.b);
		double s2 = Point::det(m.a, l.b, m.b);
		return (l.b * s1 - l.a * s2) / (s1 - s2);
	}

	static bool two_segments_intersect (Line l, Line m) {
		double dla = Point::det(l.b, m.a, m.b);
		double dlb = Point::det(l.a, m.a, m.b);
		double dma = Point::det(m.b, l.a, l.b);
		double dmb = Point::det(m.a, l.a, l.b);
		if (sign(dla * dlb) == -1 && sign(dma * dmb) == -1) return true;
		if (sign(dla) == 0 && on_segment(m, l.b)) return true;
		if (sign(dlb) == 0 && on_segment(m, l.a)) return true;
		if (sign(dma) == 0 && on_segment(l, m.b)) return true;
		if (sign(dmb) == 0 && on_segment(l, m.a)) return true;
		return false;
	}

	static bool any_segments_intersect (vector<Line> ls) {
		vector<pair<Point,pair<int,int>>> items;
		for (int i = 0; i < ls.size(); i++) {
			Line &l = ls[i];
			if (l.a.x > l.b.x) swap(l.a, l.b);
			items.push_back({l.a, {0, i}});
			items.push_back({l.b, {1, i}});
		}
		sort(items.begin(), items.end(), [](pair<Point,pair<int,int>> a, pair<Point,pair<int,int>> b) -> bool {
			if (sign(a.first.x - b.first.x) == -1) return true;
			if (sign(a.first.x - b.first.x) == 1) return false;
			if (a.second.first < b.second.first) return true;
			if (a.second.first > b.second.first) return false;
			return a.first.y < b.first.y;
		});
		auto cmp = [&](int i, int j) -> bool { return ls[i].a.y < ls[j].a.y; };
		set<int,decltype(cmp)> s(cmp);
		for (auto &item : items) {
			if (item.second.first == 0) {
				auto it = s.insert(item.second.second).first;
				int id = *it;
				int prev_id = (it == s.begin()) ? -1 : *(prev(it));
				int next_id = (next(it) == s.end()) ? -1 : *(next(it));
				if (prev_id != -1 && two_segments_intersect(ls[id], ls[prev_id])) return true;
				if (next_id != -1 && two_segments_intersect(ls[id], ls[next_id])) return true;
			} else {
				auto it = s.find(item.second.second);
				int id = *it;
				int prev_id = (it == s.begin()) ? -1 : *(prev(it));
				int next_id = (next(it) == s.end()) ? -1 : *(next(it));
				if (prev_id != -1 && next_id != -1 && two_segments_intersect(ls[prev_id], ls[next_id])) return true;
				s.erase(it);
			}
		}
		return false;
	}
};
\end{lstlisting}
\subsection{Halfplane}
\begin{lstlisting}
struct HalfPlane {
	Point s, t;  // half plane on the left of ray from p to q
	HalfPlane (Point s, Point t) : s(s), t(t) {}

	double eval (Point p) {
		double a, b, c;  // ax+by+c<=0
		a = t.y - s.y;
		b = s.x - t.x;
		c = Point::cross(t, s);
		return p.x * a + p.y * b + c;
	}

	static Point halfplane_x_line (HalfPlane hp, Line l) {
		Point p = l.a, q = l.b;
		double vp = hp.eval(p), vq = hp.eval(q);
		double x = (vq * p.x - vp * q.x) / (vq - vp);
		double y = (vq * p.y - vp * q.y) / (vq - vp);
		return Point(x, y);
	}

	static vector<Point> halfplanes_x (vector<HalfPlane> hps) {
		sort(hps.begin(), hps.end(), [](HalfPlane a, HalfPlane b) -> bool {
			int sgn = sign(Point::arg(a.t - a.s) - Point::arg(b.t - b.s));
			return sgn == 0 ? (sign(b.eval(a.s)) == -1) : (sgn < 0);
		});
		deque<HalfPlane> q {hps[0]};
		deque<Point> ans;
		for (int i = 1; i < hps.size(); i++) {
			if (sign(Point::arg(hps[i].t - hps[i].s) - Point::arg(hps[i-1].t - hps[i-1].s)) == 0) continue;
			while (ans.size() > 0 && sign(hps[i].eval(ans.back())) == 1) { ans.pop_back(); q.pop_back(); }
			while (ans.size() > 0 && sign(hps[i].eval(ans.front())) == 1) { ans.pop_front(); q.pop_front(); }
			ans.push_back(Line::line_x_line(Line(q.back().s, q.back().t), Line(hps[i].s, hps[i].t)));
			q.push_back(hps[i]);
		}
		while (ans.size() > 0 && sign(q.front().eval(ans.back())) == 1) { ans.pop_back(); q.pop_back(); }
		while (ans.size() > 0 && sign(q.back().eval(ans.front())) == 1) { ans.pop_front(); q.pop_front(); }
		ans.push_back(Line::line_x_line(Line(q.back().s, q.back().t), Line(q.front().s, q.front().t)));
		return vector<Point>(ans.begin(), ans.end());
	}
};
\end{lstlisting}
\subsection{Polygon}
\begin{lstlisting}
struct Polygon {
	int n;
	vector<Point> p;  // always counter-clockwise

	Polygon (vector<Point> p) : p(p), n(p.size()) {}

	double perimeter () {
		double ans = 0;
		for (int i = 0; i < n; i++) {
			ans += Point::dist(p[i], p[(i+1)%n]);
		}
		return ans;
	}

	double area () {
		double ans = 0;
		for (int i = 1; i < n-1; i++) {
			ans += Point::det(p[0], p[i], p[i+1]) / 2.0;
		}
		return ans;
	}

	Point mass_center () {
		Point ans(0.0, 0.0);
		double a = area();
		if (sign(a) == 0) return ans;
		for (int i = 1; i < n-1; i++) {
			ans = ans + ((p[0] + p[i] + p[i+1]) / 3.0) * (Point::det(p[0], p[i], p[i+1]) / 2.0);
		}
		return ans / a;
	}

	// first is grid point inside polygon; second is grid point on edge. vertices has to be grid points
	pair<int,int> grid_point_cnt () {
		int first = 0, second = 0;
		for (int i = 0; i < n; i++) {
			second += gcd(abs((int)(p[(i+1)%n].x - p[i].x)), abs((int)(p[(i+1)%n].y - p[i].y)));
		}
		first = (int)area() + 1 - second / 2;
		return {first, second};
	}
	int gcd(int p, int q) { return q == 0 ? p : gcd(q, p%q); }

	bool is_simple_convex_polygon () {
		for (int i = 0; i < n; i++) {  // convexity
			if (sign(Point::det(p[i], p[(i+1)%n], p[(i+2)%n])) == -1) return false;
		}
		for (int i = 1; i < n-1; i++) {  // simplicity
			if (sign(Point::det(p[0], p[i], p[i+1])) == -1) return false;
		}
		return true;
	}

	// O(n)
	// returns 1 for in, 0 for on, -1 for out
	static int point_in_polygon (Polygon po, Point p0) {
		int cnt = 0;
		for (int i = 0; i < po.n; i++) {
			if (Line::on_segment(Line(po.p[i], po.p[(i+1)%po.n]), p0)) return 0;
			int k = sign(Point::det(p0, po.p[i], po.p[(i+1)%po.n]));
			int d1 = sign(po.p[i].y - p0.y);
			int d2 = sign(po.p[(i+1)%po.n].y - p0.y);
			if (k == 1 && d1 != 1 && d2 == 1) cnt++;
			if (k == -1 && d2 != 1 && d1 == 1) cnt--;
		}
		return cnt ? 1 : -1;
	}

	// O(log(n))
	// returns 1 for in, 0 for on, -1 for out
	static int point_in_convex_polygon (Polygon po, Point p0) {
		Point point = (po.p[0] + po.p[po.n/3] + po.p[2*po.n/3]) / 3.0;
		int l = 0, r = po.n;
		while (r - l > 1) {
			int mid = (l + r) / 2;
			if (sign(Point::det(point, po.p[l], po.p[mid])) == 1) {
				if (sign(Point::det(point, po.p[l], p0)) != -1 && sign(Point::det(point, po.p[mid], p0)) == -1) r = mid;
				else l = mid;
			} else {
				if (sign(Point::det(point, po.p[l], p0)) == -1 && sign(Point::det(point, po.p[mid], p0)) != -1) l = mid;
				else r = mid;
			}
		}
		r %= po.n;
		return -sign(Point::det(p0, po.p[r], po.p[l]));
	}

	Polygon convex_polygon_x_halfplane (HalfPlane hp, Polygon po) {
		vector<Point> ps;
		for (int i = 0; i < po.n; i++) {
			if (sign(hp.eval(po.p[i])) == -1) {
				ps.push_back(po.p[i]);
			} else {
				if (sign(hp.eval(po.p[(i-1+po.n)%po.n])) == -1) {
					ps.push_back(HalfPlane::halfplane_x_line(hp, Line(po.p[i], po.p[(i-1+po.n)%po.n])));
				}
				if (sign(hp.eval(po.p[(i+1)%po.n])) == -1) {
					ps.push_back(HalfPlane::halfplane_x_line(hp, Line(po.p[i], po.p[(i+1)%po.n])));
				}
			}
		}
		return Polygon(ps);
	}

	static Polygon convex_polygon_x_convex_polygon (Polygon po1, Polygon po2) {
		vector<HalfPlane> hps;
		for (int i = 0; i < po1.n; i++) {
			hps.push_back(HalfPlane(po1.p[i], po1.p[(i+1)%po1.n]));
		}
		for (int i = 0; i < po2.n; i++) {
			hps.push_back(HalfPlane(po2.p[i], po2.p[(i+1)%po2.n]));
		}
		return Polygon(HalfPlane::halfplanes_x(hps));
	}
};
\end{lstlisting}
\subsection{Circle}
\begin{lstlisting}
struct Circle {
	Point center;
	double radius;

	Circle (Point center, double radius) : center(center), radius(radius) {}

	static bool in_circle (Circle c, Point p) {
		return sign(Point::dist(p, c.center) - c.radius) == -1;
	}

	static Circle min_circle_cover (vector<Point> p) {
		Circle ans(p[0], 0.0);
		random_shuffle(p.begin(), p.end());
		for (int i = 1; i < p.size(); i++) if (!in_circle(ans, p[i])) {
			ans.center = p[i]; ans.radius = 0;
			for (int j = 0; j < i; j++) if (!in_circle(ans, p[j])) {
				ans.center = Point::outer_center(p[i], p[j]);
				ans.radius = Point::dist(p[j], ans.center);
				for (int k = 0; k < j; k++) if (!in_circle(ans, p[k])) {
					ans.center = Point::outer_center(p[i], p[j], p[k]);
					ans.radius = Point::dist(p[k], ans.center);
				}
			}
		}
		return ans;
	}
};
\end{lstlisting}
\subsection{Simplex volume}
\begin{lstlisting}
// AB AC AD BC BD CD
double simplex_volume (double l, double n, double a, double m, double b, double c) {
	double x = 4*a*a*b*b*c*c - a*a*(b*b+c*c-m*m)*(b*b+c*c-m*m) - b*b*(c*c+a*a-n*n)*(c*c+a*a-n*n);
	double y = c*c*(a*a+b*b-l*l)*(a*a+b*b-l*l) - (a*a+b*b-l*l)*(b*b+c*c-m*m)*(c*c+a*a-n*n);
	return sqrt(x-y) / 12;
}
\end{lstlisting}
\subsection{Count gridpoints under a line}
\begin{lstlisting}
// Count gridpoints under a line
// Compute for (int i = 0; i < n; i++) s += floor((a+b*i)/m);
long long count_gridpoints (long long n, long long a, long long b, long long m) {
  if (b == 0) return n * (a / m);
  if (a >= m) return n * (a / m) + count_gridpoints(n, a%m, b, m);
  if (b >= m) return (n-1) * n / 2 * (b / m) + count_gridpoints(n, a, b%m, m);
  return count_gridpoints((a+b*n)/m, (a+b*n)%m, m, b);
}
\end{lstlisting}
\subsection{Simpson's Union Of Circles}
\begin{lstlisting}
struct dot
{
    double x, y;
    double dis(dot &o)
    {
        return sqrt(sqr(x - o.x) + sqr(y - o.y));
    }
};
int lx = 1000, rx = -1000;
struct circle
{
    dot o; int r;
    void init()
    {
        int x, y;
        scanf("%d%d%d", &x, &y, &r);
        lx = min(lx, x - r); rx = max(rx, x + r);
        o.x = x; o.y = y;
    }
    bool in(circle &b)
    {
        return (b.r - r - o.dis(b.o) >= -eps);
    }
    bool operator==(const circle &b)
    {
        return r == b.r && fabs(o.x - b.o.x) <= eps && fabs(o.y - b.o.y) <= eps;
    }
}tmp[Maxn], c[Maxn];
struct seg
{
    double v; int s;
    bool operator<(const seg &o)
        const{return v < o.v - eps;}
}l[Maxn * 2];
int n, m;

void Init()
{
    scanf("%d", &m);
    for (int i = 1; i <= m; ++i)
    {
        tmp[++n].init();
        for (int j = 1; j <= n - 1; ++j)
            if (tmp[j] == tmp[n])
                {--n; break;}
    }
    m = n; n = 0;
    for (int i = 1; i <= m; ++i)
    {
        bool f = 0;
        for (int j = 1; j <= m; ++j) if (j != i)
            if (tmp[i].in(tmp[j]))
            {
                f = 1;
                break;
            }
        if (!f) c[++n] = tmp[i];
    }
}

inline double get(double x)
{
    int t = 0, now = 0;
    double d, last, s = 0;
    for (int i = 1; i <= n; ++i)
    {
        if (fabs(x - c[i].o.x) - c[i].r >= -eps) continue;
        d = sqrt(sqr(c[i].r) - sqr(x - c[i].o.x));
        l[++t].v = c[i].o.y - d; l[t].s = 1;
        l[++t].v = c[i].o.y + d; l[t].s = -1;
    }
    sort(l + 1, l + 1 + t);
    for (int i = 1; i <= t; ++i)
    {
        now += l[i].s;
        if (now == 1 && l[i].s == 1) last = l[i].v;
        if (now == 0) s += l[i].v - last;
    }
    return s;
}

double simpson(double l, double r, double lx, double mx, double rx)
{
    double m = (l + r) * 0.5, lp, rp, s, ls, rs;
    lp = get((l + m) * 0.5);
    rp = get((m + r) * 0.5);
    s = (lx + rx + 4 * mx) * (r - l) / 6;
    ls = (lx + mx + 4 * lp) * (m - l) / 6;
    rs = (mx + rx + 4 * rp) * (r - m) / 6;
    if (fabs(ls + rs - s) <= 1e-6)
        return s;
    return simpson(l, m, lx, lp, mx) + simpson(m, r, mx, rp, rx);
}

void Work()
{
    double s = 0, last = get(lx), now;
    for (int i = lx; i <= rx - 1; ++i)
    {
        now = get(i + 1);
        if (fabs(last) > eps || fabs(now) > eps)
            s += simpson(i, i + 1, last, get(i + 0.5), now);
        last = now;
    }
    printf("%.3lf\n", s);
}
\end{lstlisting}
\section{Misc}
\subsection{Date}
\subsubsection{Date to Day of Week}
\begin{lstlisting}
int whatday (int d, int m, int y) {
  int ans;
  if (m == 1 || m == 2) { m += 12; y--; }
  if (y < 1752 || (y == 1752 && m < 9) || (y == 1752 && m == 9 && d < 3)) {
    ans = (d + 2*m + 3*(m+1)/5 + y + y/4+5) % 7;
  } else {
    ans = (d + 2*m + 3*(m+1)/5 + y + y/4 - y/100 + y/400) % 7;
  }
  return ans;
}
\end{lstlisting}
\subsubsection{Count Days from AD}
\begin{lstlisting}
const int days = 365;
const int s[] = {0, 31, 28, 31, 30, 31, 30, 31, 31, 30, 31, 30, 31};
bool IsLeap (int y) {
  return y % 400 == 0 || (y % 100 && y % 4 == 0);
}
int leap (int y) {
  return y/4 - y/100 + y/400;
}
int calc (int day, int mon, int year) {
  int res = (year-1) * days + leap(year-1);
  for (int i = 1; i < mon; ++i) res += s[i];
  if (IsLeap(year) && mon > 2) res++;
  res += day;
  return res;
}
\end{lstlisting}

\end{multicols}
\end{document}
